\documentclass[a4paper,10pt,BCOR10mm,oneside,headsepline]{scrartcl}
\usepackage[english]{babel}
\usepackage[utf8]{inputenc}
\usepackage{wasysym}% provides \ocircle and \Box
\usepackage{enumitem}% easy control of topsep and leftmargin for lists
\usepackage{color}% used for background color
\usepackage{forloop}% used for \Qrating and \Qlines
\usepackage{ifthen}% used for \Qitem and \QItem
\usepackage{typearea}
\areaset{17cm}{26cm}
\setlength{\topmargin}{-1cm}
\usepackage{scrpage2}
\pagestyle{scrheadings}
\ihead{code quality checking questionnaire}
\ohead{\pagemark}
\chead{}
\cfoot{}

%%%%%%%%%%%%%%%%%%%%%%%%%%%%%%%%%%%%%%%%%%%%%%%%%%%%%%%%%%%%
%% Beginning of questionnaire command definitions %%
%%%%%%%%%%%%%%%%%%%%%%%%%%%%%%%%%%%%%%%%%%%%%%%%%%%%%%%%%%%%
%%
%% 2010, 2012 by Sven Hartenstein
%% mail@svenhartenstein.de
%% http://www.svenhartenstein.de
%%
%% Please be warned that this is NOT a full-featured framework for
%% creating (all sorts of) questionnaires. Rather, it is a small
%% collection of LaTeX commands that I found useful when creating a
%% questionnaire. Feel free to copy and adjust any parts you like.
%% Most probably, you will want to change the commands, so that they
%% fit your taste.
%%
%% Also note that I am not a LaTeX expert! Things can very likely be
%% done much more elegant than I was able to. If you have suggestions
%% about what can be improved please send me an email. I intend to
%% add good tipps to my website and to name contributers of course.
%%
%% 10/2012: Thanks to karathan for the suggestion to put \noindent
%% before \rule!

%% \Qq = Questionaire question. Oh, this is just too simple. It helps
%% making it easy to globally change the appearance of questions.
\newcommand{\Qq}[1]{\textbf{#1}}

%% \QO = Circle or box to be ticked. Used both by direct call and by
%% \Qrating and \Qlist.
\newcommand{\QO}{$\Box$}% or: $\ocircle$

%% \Qrating = Automatically create a rating scale with NUM steps, like
%% this: 0--0--0--0--0.
\newcounter{qr}
\newcommand{\Qrating}[1]{\QO\forloop{qr}{1}{\value{qr} < #1}{---\QO}}

%% \Qline = Again, this is very simple. It helps setting the line
%% thickness globally. Used both by direct call and by \Qlines.
\newcommand{\Qline}[1]{\noindent\rule{#1}{0.6pt}}

%% \Qlines = Insert NUM lines with width=\linewith. You can change the
%% \vskip value to adjust the spacing.
\newcounter{ql}
\newcommand{\Qlines}[1]{\forloop{ql}{0}{\value{ql}<#1}{\vskip0em\Qline{\linewidth}}}

%% \Qlist = This is an environment very similar to itemize but with
%% \QO in front of each list item. Useful for classical multiple
%% choice. Change leftmargin and topsep accourding to your taste.
\newenvironment{Qlist}{%
\renewcommand{\labelitemi}{\QO}
\begin{itemize}[leftmargin=1.5em,topsep=-.5em]
}{%
\end{itemize}
}

%% \Qtab = A "tabulator simulation". The first argument is the
%% distance from the left margin. The second argument is content which
%% is indented within the current row.
\newlength{\qt}
\newcommand{\Qtab}[2]{
\setlength{\qt}{\linewidth}
\addtolength{\qt}{-#1}
\hfill\parbox[t]{\qt}{\raggedright #2}
}

%% \Qitem = Item with automatic numbering. The first optional argument
%% can be used to create sub-items like 2a, 2b, 2c, ... The item
%% number is increased if the first argument is omitted or equals 'a'.
%% You will have to adjust this if you prefer a different numbering
%% scheme. Adjust topsep and leftmargin as needed.
\newcounter{itemnummer}
\newcommand{\Qitem}[2][]{% #1 optional, #2 notwendig
\ifthenelse{\equal{#1}{}}{\stepcounter{itemnummer}}{}
\ifthenelse{\equal{#1}{a}}{\stepcounter{itemnummer}}{}
\begin{enumerate}[topsep=2pt,leftmargin=2.8em]
\item[\textbf{\arabic{itemnummer}#1.}] #2
\end{enumerate}
}

%% \QItem = Like \Qitem but with alternating background color. This
%% might be error prone as I hard-coded some lengths (-5.25pt and
%% -3pt)! I do not yet understand why I need them.
\definecolor{bgodd}{rgb}{0.8,0.8,0.8}
\definecolor{bgeven}{rgb}{0.9,0.9,0.9}
\newcounter{itemoddeven}
\newlength{\gb}
\newcommand{\QItem}[2][]{% #1 optional, #2 notwendig
\setlength{\gb}{\linewidth}
\addtolength{\gb}{-5.25pt}
\ifthenelse{\equal{\value{itemoddeven}}{0}}{%
\noindent\colorbox{bgeven}{\hskip-3pt\begin{minipage}{\gb}\Qitem[#1]{#2}\end{minipage}}%
\stepcounter{itemoddeven}%
}{%
\noindent\colorbox{bgodd}{\hskip-3pt\begin{minipage}{\gb}\Qitem[#1]{#2}\end{minipage}}%
\setcounter{itemoddeven}{0}%
}
}

%%%%%%%%%%%%%%%%%%%%%%%%%%%%%%%%%%%%%%%%%%%%%%%%%%%%%%%%%%%%
%% End of questionnaire command definitions %%
%%%%%%%%%%%%%%%%%%%%%%%%%%%%%%%%%%%%%%%%%%%%%%%%%%%%%%%%%%%%

% \Qitem[b]{ \Qq{rating?} \Qtab{3cm}{horrible \Qrating{5} fantastic}}
% 
% \Qitem{ \Qq{checkbox with field} \hskip0.4cm \QO{}
% absolutely \hskip0.5cm \QO{} not really because: \Qline{3cm} }
% 
% \Qitem{\Qq{yes no} \Qtab{5.5cm}{\QO{} Yes
% \hskip0.5cm \QO{} No}}





\begin{document}

\begin{center}
\textbf{\huge Code Evaluation questionnaire }
\end{center}\vskip1em

\noindent this document aims to provide a guideline how to evaluate (R) code in my course

\noindent \textit{Please note: not all item might be applicable to the code you review - please cross-out any non-relevant part.}

\bigskip

\Qitem{ \Qq{Informative naming of the script?} \hskip0.4cm \QO{}
absolutely \hskip0.5cm \QO{} not really because: \Qline{3cm} }


\section*{Script Meta-Information}

\Qitem{\Qq{Information is provided at top of the script?} \Qtab{5.5cm}{\QO{} Yes
\hskip0.5cm \QO{} No}}

\Qitem{ \Qq{Authors name}: \Qline{8cm} }

% \Qitem{ \Qq{When was the script created?} Date on the script is \Qline{1.5cm}}

\Qitem{\Qq{Contact details are provided (email, url, git)?} \Qtab{5.5cm}{\QO{} Yes
\hskip0.5cm \QO{} No}}

\Qitem{\Qq{Date of development is listed?} \Qtab{5.5cm}{\QO{} Yes
\hskip0.5cm \QO{} No}}

\Qitem{ \Qq{Main purpose of the script is explained?} \hskip0.4cm \QO{}
yes \hskip0.5cm \QO{} not really because: \Qline{3cm} }

\Qitem{ \Qq{Needed input is defined?} \hskip0.4cm \QO{}
yes \hskip0.5cm \QO{} not really because: \Qline{3cm} }

\Qitem{ \Qq{Output is defined?} \hskip0.4cm \QO{}
yes \hskip0.5cm \QO{} not really because: \Qline{3cm} }

\Qitem{ \Qq{R version and R packages needed are listed?} \hskip0.4cm \QO{}
yes \hskip0.5cm \QO{} not really because: \Qline{3cm} }

\Qitem{ \Qq{Required other software is explained?} \hskip0.4cm \QO{}
yes \hskip0.5cm \QO{} not really because: \Qline{3cm} }

\Qitem{ \Qq{Informative header is well formatted?} \hskip0.4cm \QO{}
yes \hskip0.5cm \QO{} not really because: \Qline{3cm} }

\Qitem{ \Qq{All necessary details provided?}
\begin{Qlist}
\item Yes, I understand its aim and needed input
\item No, I need to check the script carefully
\item just some parts are provided.
\end{Qlist}
}

\Qitem[a]{ \Qq{What do you think until now what the output/results will be? Describe it briefly:} \Qlines{3} }




% \Qitem[d]{ \Qq{Input and Output definitions}\Qtab{2.5cm}{\parbox[t]{3.3cm}{\raggedleft
% horrible}
% \Qrating{7} \parbox[t]{3cm}{fantastic}}}
% 
% \Qitem[e]{ \Qq{xxx}\Qtab{2.5cm}{\parbox[t]{3.3cm}{\raggedleft horrible}
% \Qrating{7} \parbox[t]{3cm}{fantastic}}}

% ############################
% ############################

\bigskip 

% \noindent\makebox[\linewidth]{\rule{\paperwidth}{0.4pt}}

\noindent\rule{12cm}{0.8pt}


\section*{Actual Code for the Analysis}

\Qitem[b]{ \Qq{Well commented?} \Qtab{3cm}{horrible \Qrating{5} fantastic} remarks: \Qline{12cm}}

\Qitem[b]{ \Qq{Ratio of Comments vs. Code?} \Qtab{3cm}{horrible \Qrating{5} fantastic}}

\Qitem{ \Qq{The code can be redone without any fixes?} \hskip0.4cm \QO{}
absolutely \hskip0.5cm \QO{} not really because: \Qline{3cm} }

\Qitem[b]{ \Qq{Data can be retrieved without contact the author?}  \hskip0.4cm \QO{}
absolutely \hskip0.5cm \QO{} not really because: \Qline{3cm} }

\Qitem[b]{ \Qq{Data import is generic (no full paths, direct import possible)?} \Qtab{3cm}{yes \Qrating{5} no}}

\Qitem{\Qq{Code follows a logical structure?} \hskip0.4cm \QO{}
absolutely \hskip0.5cm \QO{} not really because: \Qline{3cm} }

\Qitem{\Qq{Code only includes useful information?} \hskip0.4cm \QO{}
absolutely \hskip0.5cm \QO{} not really because: \Qline{3cm} }

\Qitem{\Qq{Script can be run easily on other data sets (generic code)?} \hskip0.4cm \QO{}
absolutely \hskip0.5cm \QO{} not really because: \Qline{3cm} }

\Qitem{\Qq{Appropriate use of commands - no unnecessary detours?} \hskip0.4cm \QO{}
absolutely \hskip0.5cm \QO{} not really because: \Qline{3cm} }

\Qitem[b]{ \Qq{Novel code not covered in the course is used?} \Qtab{3cm}{a lot \Qrating{5} just known commands}}

\Qitem[b]{ \Qq{The script triggered interest and you learned new things?} \Qtab{3cm}{yes, a lot \Qrating{5} no, not a bit}}


\Qitem[a]{ \Qq{Please describe what was special/interesting:} \Qlines{2} }

\Qitem[b]{ \Qq{Script is fast (based on peformance measures)} \Qtab{3cm}{yes \Qrating{5} no}}

\Qitem[a]{ \Qq{Which parts should be improved?} \Qlines{2} }


\Qitem[b]{ \Qq{Data handling and transformation is coherent and well commented?} \Qtab{3cm}{yes \Qrating{5} no}}

\Qitem[a]{ \Qq{What is missing?} \Qlines{2} }


\section*{Graphs and Maps}

\Qitem{ \Qq{Graphs or Maps are providing key messages?} \hskip0.4cm \QO{}
absolutely \hskip0.5cm \QO{} not really because: \Qline{3cm} }

\Qitem{ \Qq{Plots/Maps are are self-explanatory?} \hskip0.4cm \QO{}
absolutely \hskip0.5cm \QO{} not really because: \Qline{3cm} }

\Qitem{ \Qq{Graphs or Maps include all necessary items?} \hskip0.4cm \QO{}
absolutely \hskip0.5cm \QO{} not really because: \Qline{3cm} }

\Qitem[b]{ \Qq{Plots/maps are not overloaded?} \Qtab{3cm}{yes, clean \Qrating{5} no, totally cluttered}}

\Qitem[b]{ \Qq{Plots/maps are informative?} \Qtab{3cm}{yes \Qrating{5} no}}

\Qitem[b]{ \Qq{Plots/maps layout is consistent?}  \hskip0.4cm \QO{}
absolutely \hskip0.5cm \QO{} not really because: \Qline{3cm} }

\Qitem[b]{ \Qq{Plots/maps have appropriate colour scheme?}  \hskip0.4cm \QO{}
absolutely \hskip0.5cm \QO{} not really because: \Qline{3cm} }

\Qitem[b]{ \Qq{Maps have scale bars, legend, coordinates?}  \hskip0.4cm \QO{}
absolutely \hskip0.5cm \QO{} not really because: \Qline{3cm} }

\Qitem[b]{ \Qq{Maps include landmarks, cities, roads for orientation?}  \hskip0.4cm \QO{}
absolutely \hskip0.5cm \QO{} not really because: \Qline{3cm} }


\section*{Overall Impression}

\minisec{Please evaluate the following parts}
\vskip.5em

\Qitem[a]{ \Qq{Readability} \Qtab{3cm}{horrible \Qrating{5} fantastic}}

\Qitem[b]{ \Qq{Information} \Qtab{3cm}{horrible \Qrating{5} fantastic}}

\Qitem[c]{ \Qq{Aim} \Qtab{3cm}{horrible \Qrating{5}
fantastic}}

\bigskip

\Qitem{ \Qq{Do you think it qualifies for being scientifically reproducible?}
\begin{Qlist}
\item yes
\item no
\item needs some more work
\item details: \Qline{4cm}
\end{Qlist}
}


\bigskip

\Qitem{\Qq{Is the script really worth the effort?} \Qtab{5.5cm}{\QO{} Yes, totally.
\hskip0.5cm \QO{} Probably not. \hskip0.5cm \QO{} Don't know.}}

\Qitem{\Qq{Would you be interested to use this script for your analysis?}

\Qtab{10.5cm}{\QO{}\Qtab{0.6cm}{yes, would love to}}\par
\Qtab{10.5cm}{\QO{}\Qtab{0.6cm}{No clue. I just can't figure it out.
So sorry.}}\par
\Qtab{10.5cm}{\QO{}\Qtab{0.6cm}{yes, definitely parts of it.}}\par
}

\Qitem[a]{ \Qq{Please describe your impression.} \Qlines{3} }

\Qitem[a]{ \Qq{When you check your anticipated results/output (Q 13) at the beginning - where you expectations met? and if no, why not:} \Qlines{3} }


\Qitem[b]{ \Qq{What do you especially like about this script:} \Qlines{5} }

\Qitem[b]{ \Qq{What do you especially \underline{dislike} about this script:} \Qlines{5} }


\end{document}
